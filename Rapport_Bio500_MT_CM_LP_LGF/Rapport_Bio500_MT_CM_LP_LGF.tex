\documentclass[9pt,twocolumn,twoside,]{pnas-new}

% Use the lineno option to display guide line numbers if required.
% Note that the use of elements such as single-column equations
% may affect the guide line number alignment.


\usepackage[T1]{fontenc}
\usepackage[utf8]{inputenc}

% tightlist command for lists without linebreak
\providecommand{\tightlist}{%
  \setlength{\itemsep}{0pt}\setlength{\parskip}{0pt}}




\templatetype{pnasresearcharticle}  % Choose template

\title{Évolution de la diversité de Lépidoptères selon un gradient
spatial et temporel}

\author[]{Clovis Marion}
\author[]{Léane Plouffe}
\author[]{Laurent Fournelle-Grenier}
\author[]{Mariève Trottier}



% Please give the surname of the lead author for the running footer
\leadauthor{}

% Please add here a significance statement to explain the relevance of your work
\significancestatement{}


\authorcontributions{}



\correspondingauthor{\textsuperscript{} }

% Keywords are not mandatory, but authors are strongly encouraged to provide them. If provided, please include two to five keywords, separated by the pipe symbol, e.g:
 \keywords{  Lépidoptère |  iNaturalist |  Science citoyenne |  Zones
climatiques |  Changements climatiques  } 

\begin{abstract}
Dans un contexte de changements environnementaux rapides, les structures
des communautés d'insectes, dont les lépidoptères, subissent
d'importantes transformations. Cette étude s'appuie sur des données,
provenant en partie d'observations citoyennes, pour analyser la
distribution spatiale des lépidoptères au Québec. La richesse spécifique
est examinée selon la latitude et la longitude, afin de détecter des
tendances spatiales liées aux zones climatiques du territoire québécois.
\end{abstract}

\dates{This manuscript was compiled on \today}
\doi{\url{www.pnas.org/cgi/doi/10.1073/pnas.XXXXXXXXXX}}

\begin{document}

% Optional adjustment to line up main text (after abstract) of first page with line numbers, when using both lineno and twocolumn options.
% You should only change this length when you've finalised the article contents.
\verticaladjustment{-2pt}



\maketitle
\thispagestyle{firststyle}
\ifthenelse{\boolean{shortarticle}}{\ifthenelse{\boolean{singlecolumn}}{\abscontentformatted}{\abscontent}}{}

% If your first paragraph (i.e. with the \dropcap) contains a list environment (quote, quotation, theorem, definition, enumerate, itemize...), the line after the list may have some extra indentation. If this is the case, add \parshape=0 to the end of the list environment.

\acknow{}

\section*{Méthodologie}\label{muxe9thodologie}
\addcontentsline{toc}{section}{Méthodologie}

Mettre la métho ici

\section*{Résultats}\label{ruxe9sultats}
\addcontentsline{toc}{section}{Résultats}

Mettre résultats ici

\section*{Discussion}\label{discussion}
\addcontentsline{toc}{section}{Discussion}

Cela dit, on remarque qu'il y a moins d'observations vers l'est de la
province. Cette diminution pourrait s'expliquer par la présence de
grandes étendues d'eau qui traversent le territoire à cette latitude, ce
qui pourrait limiter non seulement la présence d'espèces, mais aussi les
efforts de prospection dans ces zones (source).

\showmatmethods
\showacknow
\pnasbreak



% Bibliography
% \bibliography{pnas-sample}

\end{document}
